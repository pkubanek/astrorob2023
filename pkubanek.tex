\documentclass[proceedings, preprint]{rmaa}

% The preprint option sets the first page header to contain the name
% of the conference. It will be ignored when typesetting the final
% volume. 

%%%
%%% Load any optional packages you need here with \usepackage
%%% 

% This allows compact, in-paragraph, and as-paragraph  versions of the
% standard itemize and enumerate environments. 
\usepackage{paralist}

% These are used in one of the graphics examples
\usepackage{psfrag,color}

%%%
%%% Define any personal macros here
%%% 

% These are some I use in typesetting example code
\newcommand{\bs}{\textbackslash}
\newcommand{\CS}[1]{\texttt{\textbackslash #1}}
\newenvironment{Example}
{\begin{list}{}{\setlength{\leftmargin}{5pt}\setlength{\rightmargin}{5pt}}\item[]}
  {\end{list}}

  
%%%
%%% Article preamble commands (title, authors, abstract, etc.) 
%%% None of these produce any output themselves, they just set things 
%%% up for \maketitle
%%%

% This is only used for making the header for the preprint version
\SetYear{2023}
\SetConfTitle{Astrorob 2023i Proceedings}

% Please use mixed case here, since this title gets propagated onto
% the web page, ADS entry, etc. 
\title{LSST Control System and Synergies Between LSST and Small Observatories} 

% For the conference proceedings, the author affiliations should be
% subscripted, using \altaffil and/or \altaffilmark + \altaffiltext
% Note that \altaffilmark goes after a comma and that `and' is spelt
% out.
\author{
  P. Kub\'anek,\altaffilmark{1} 
}

% Note that \altaffil, \altaffilmark go inside the scope of the
% \author{...} command but \altaffiltext is outside it. 
\altaffiltext{1}{Vera C. Rubin Observatory,
  1500 Av. Ju\'an Cisternas,
  170000, La Serena, Chile}

% Authors for running headers - surnames only, et al. if more than 3. 
\shortauthor{Kub\'anek}
% Title for running header
\shorttitle{LSST Control System}

% List of authors used to construct table of contents
\listofauthors{P. Kub\'anek}
% Each author in Surname, Initials format, used in generating Author
% Index entries.
\indexauthor{Kub\'anek, P.}

% English abstract
\abstract{This document gives overview of the LSST Control System. The control
system is set to operate telescopes and facilities of the constructed Vera C.
Rubin observatory telescope. The system relies heavily on using Kubernetes
cluster management, running Dockerized containers. Each device is controled by
a small control daemon, called CSC - Commandable SAL Component (SAL = System
Access Layer). OpenDDS is used as a communication middleware, enabling CSC
communication.}

% Spanish abstract - leave blank and it will be translated by the
% editors. 
%\resumen{Presentamos los resultados de observaciones a frecuencias
%  m\'ultiples del proplyd LV~2 (M42~167--317) en la nebulosa de Ori\'on,
%  concentrando sobre nuestra analis\'\i{}s del ``microjet'' unidireccional
%  que sale del frente de ionizci\'on del proplyd.}


% Keywords must be from the standard list and in alphabetical order. 
% You should have no more than SIX different keywords. 
\addkeyword{LSST}
\addkeyword{Telescope Control System}
\addkeyword{Vera C. Rubin Observatory}


%%%
%%% Beginning of document proper
%%%
\begin{document}
% Typeset article header
\maketitle


\section{Introduction}
\label{sec:intro}

The Vera C. Rubin Observatory is an observatory dedicated to southern full sky
survey. It is part of the LSST\cite{2019lsst} project, which includes 8.4m
primary mirror main telescope, 1.2m auxiliary calbration telescope and a
networked data centers. It's main camera, boasting 3.2 gigapixel 189 4k x 4k
CCD sensors should take an image every 15 seconds, surveying the full visible
portion of the night sky every 3 nights.

The observatory control system is completely designed in-house.
OpenDDS\cite{opendds} was selected as the communication middleware. The
messages passed through DDS are described in a XML files\cite{salxml}.

\section{Hardware control}

\subsection{ILC - Inner Loop Controller}

The LSST utilizes ILC\cite{2014ilc} - Inner Loop Controller - in vast quantities. Over 500 of
the boards were manufactured and tested. The custom design of the boards was
chosen primary for LSST requirement to keep the electronics in the mirror cell
during its coating in vacuum.

The boards provide a standardized interface set of the digital inputs and
outputs. A custom firmware is running on the microcontroller.

The ILCs are being commanded from M1M3 controller via a standard RS-485 bus.
ILC address is provided by a chip residing in power and communication
connector, so the ILCs can be easily changed if needed.

\subsection{M1M3 control software}

The M1M3 control software main task is to ensure M1M3 glass safety. Apart from
that, it shall control M1M3 movements so its optical surfaces form a perfect
telescope. The mirror is being supported and actuated by 156 pneumatic
actuators. Mirror position is steered by 6 linear actuators - hardpoints -
forming a hexapod.

Mirror temperature is critical for proper function of the optics. Mirror
deformations due to different temperatures shall be avoid. For that, the mirror
is actively cooled using forced air injection system. The air is being cooled
by glycol heat exchanges, with the extra heat removed outside of the telescope
building.

\subsection{M1M3 Force actuators}

The force actuators use a ILC, mentioned above. The ILC provides 

\section{EFD - Engineering Facility Database}

All telemetry data acquired on the summit, together with commands send, are
stored in the Engineering Facility Database - EFD. EFD uses the InfluxDB time
series database to store data. Chronograf instance is available to visualize
stored data.

Data retried from the EFD are extensively used during observatory
commissioning.

\section{System safety}

Safety system is designed as being completely independent from the control
software. It uses industrial standard safety Programmable Logic Controllers -
PLCs to implement interlocks pertaining throughout the observatory. The
interlocks make sure equipment is stopped as soon as any, either internal or
external (earthquakes) events happen.

\section{Synergies between LSST and small observatories}

LSST shall detect about 10 millions transients per night, that's about 10
thousand per visit\cite{lsstdata}. Images shall be processed and transient information
available within 60 seconds after images are acquired.

The transient data are pre-filtered by community developed and managed
brokers\cite{databrokers}. Independent observatories shall use broker provided
services to select triggers they shall follow. Brokers add context to the
detection - catalogue cross-matches, suitability for observations.

\subsection{Observatory ecosystem}

LSST takes and process data. It's assumed to be community responsibility to
provide follow-up observations.

Due to short exposure, large collecting area and unique CCD design (high full
well depth), the magnitude range of the transients is suitable for small
telescopes (from ~6). Follow-up observations, providing more detailed light
curve, will be crucial.

Spectroscopic evolution of the sources will be interesting as well. Of course,
large diameter telescopes will be better for spectroscopy.


\begin{thebibliography}
	\bibitem[\v{Z}eljko, I. et al.(2019)]{2019lsst} "LSST: from science drivers to reference design and anticipated data products." \apj 873.2: 111.

  \bibitem[OpenDDS web site (2024)]{opendds}\url{https://opendds.org/}

  \bibitem[LSST Data Products Definition Document(2023)]{lsstdata}LSE-163 \url{https://lse-163.lsst.io/}

  \bibitem[LSST Alert Brokers(2024)]{databrokers}\url{https://www.lsst.org/scientists/alert-brokers}

  \bibitem[Dave, M., Schumacher, G. and Lotz, P.(2016)]{salxml}"LSST communications middleware implementation." Ground-based and Airborne Telescopes VI. Vol. 9906. Spie

  \bibitem[Wiecha, O. and Lotz, P.(2014)]{2014ilc}LSST Inner Loop Controller (ILC) design and characterization, Advances in Optical and Mechanical Technologies for Telescopes and Instrumentation Vol. 9151., Spie

\end{thebibliography}

\end{document}
