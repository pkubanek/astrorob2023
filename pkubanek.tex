\documentclass[proceedings, preprint]{rmaa}

% The preprint option sets the first page header to contain the name
% of the conference. It will be ignored when typesetting the final
% volume. 

%%%
%%% Load any optional packages you need here with \usepackage
%%% 

% This allows compact, in-paragraph, and as-paragraph  versions of the
% standard itemize and enumerate environments. 
\usepackage{paralist}

% These are used in one of the graphics examples
\usepackage{psfrag,color}

%%%
%%% Define any personal macros here
%%% 

% These are some I use in typesetting example code
\newcommand{\bs}{\textbackslash}
\newcommand{\CS}[1]{\texttt{\textbackslash #1}}
\newenvironment{Example}
{\begin{list}{}{\setlength{\leftmargin}{5pt}\setlength{\rightmargin}{5pt}}\item[]}
  {\end{list}}

  
%%%
%%% Article preamble commands (title, authors, abstract, etc.) 
%%% None of these produce any output themselves, they just set things 
%%% up for \maketitle
%%%

% This is only used for making the header for the preprint version
\SetYear{2023}
\SetConfTitle{Astrorob 2023i Proceedings}

% Please use mixed case here, since this title gets propagated onto
% the web page, ADS entry, etc. 
\title{LSST Control System and Synergies Between LSST and Small Observatories} 

% For the conference proceedings, the author affiliations should be
% subscripted, using \altaffil and/or \altaffilmark + \altaffiltext
% Note that \altaffilmark goes after a comma and that `and' is spelt
% out.
\author{
  P. Kub\'anek,\altaffilmark{1} 
}

% Note that \altaffil, \altaffilmark go inside the scope of the
% \author{...} command but \altaffiltext is outside it. 
\altaffiltext{1}{Vera C. Rubin Observatory,
  1500 Av. Ju\'an Cisternas,
  170000, La Serena, Chile}

% Authors for running headers - surnames only, et al. if more than 3. 
\shortauthor{Kub\'anek}
% Title for running header
\shorttitle{LSST Control System}

% List of authors used to construct table of contents
\listofauthors{P. Kub\'anek}
% Each author in Surname, Initials format, used in generating Author
% Index entries.
\indexauthor{Kub\'anek, P.}

% English abstract
\abstract{This document gives overview of the LSST Control System. The control
system is set to operate telescopes and facilities of the constructed Vera C.
Rubin observatory telescope. The system relies heavily on using Kubernetes
cluster management, running Dockerized containers. Each device is controled by
a small control daemon, called CSC - Commandable SAL Component (SAL = System
Access Layer). OpenDDS is used as a communication middleware, enabling CSC
communication.}

% Spanish abstract - leave blank and it will be translated by the
% editors. 
%\resumen{Presentamos los resultados de observaciones a frecuencias
%  m\'ultiples del proplyd LV~2 (M42~167--317) en la nebulosa de Ori\'on,
%  concentrando sobre nuestra analis\'\i{}s del ``microjet'' unidireccional
%  que sale del frente de ionizci\'on del proplyd.}


% Keywords must be from the standard list and in alphabetical order. 
% You should have no more than SIX different keywords. 
\addkeyword{LSST}
\addkeyword{Telescope Control System}
\addkeyword{Vera C. Rubin Observatory}


%%%
%%% Beginning of document proper
%%%
\begin{document}
% Typeset article header
\maketitle


\section{Introduction}
\label{sec:intro}

The Vera C. Rubin Observatory is an observatory dedicated to southern full sky
survey. It is part of the LSST\cite{2019lsst} project, which includes 8.4m
primary mirror main telescope, 1.2m auxiliary calbration telescope and a
networked data centers. It's main camera, boasting 3.2 gigapixel 189 4k x 4k
CCD sensors should take an image every 15 seconds, surveying the full visible
portion of the sky every 3 nights.

The observatory control system is completely designed in-house.
OpenDDS\cite{opendds} was selected as the communication middleware. The
messages passed through DDS are described in a XML files\cite{salxml}.

\section{M1M3 control software}

The M1M3 control software main task is to ensure M1M3 glass safety. Apart from
that, it shall control M1M3 movements so its optical surfaces form a perfect
telescope. The mirror is being supported and actuated by 156 pneumatic
actuators. Mirror position is steered by 6 linear actuators - hardpoints -
forming a hexapod.

\begin{thebibliography}
  \bibitem[Ivezi\v{c}, \v{Z}eljko, et al.(2019)]{2019lsst} "LSST: from science drivers to reference design and anticipated data products." \apj 873.2: 111.

  \bibitem[OpenDDS web site (2024)]{opendds}\url{https://opendds.org/}

  \bibitem[Mills, Dave, German Schumacher, and Paul Lotz (2016)]{salxml}"LSST communications middleware implementation." Ground-based and Airborne Telescopes VI. Vol. 9906. Spie

\end{thebibliography}


\end{document}
